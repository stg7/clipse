\documentclass{article}

\usepackage{arxiv}

\usepackage[utf8]{inputenc} % allow utf-8 input
\usepackage[T1]{fontenc}    % use 8-bit T1 fonts
\usepackage{color}
\usepackage{hyperref}
\hypersetup{colorlinks,linkcolor=blue,urlcolor=cyan}
\usepackage{url}            % simple URL typesetting
\usepackage{booktabs}       % professional-quality tables
\usepackage{amsfonts}       % blackboard math symbols
\usepackage{nicefrac}       % compact symbols for 1/2, etc.
\usepackage{microtype}      % microtypography
\usepackage{cleveref}       % smart cross-referencing
\usepackage{graphicx}
\usepackage{natbib}
\usepackage{doi}
\graphicspath{{./imgs/}}

\title{CLIPSE -- a minimalistic CLIP-based image search engine for research}

% Here you can change the date presented in the paper title
%\date{September 9, 1985}
% Or remove it
%\date{}

\newif\ifuniqueAffiliation
% Comment to use multiple affiliations variant of author block
\uniqueAffiliationtrue


\author{ Steve Göring\hspace{1mm}\href{https://orcid.org/0000-0001-6810-6969}{\includegraphics[scale=0.06]{orcid.pdf}}\\
    Audiovisual Technology Group\\
    Technische Universität Ilmenau\\
    Germany \\
    \texttt{steve.goering@tu-ilmenau.de} \\
}


% Uncomment to override  the `A preprint' in the header
%\renewcommand{\headeright}{Technical Report}
%\renewcommand{\undertitle}{Technical Report}
\renewcommand{\shorttitle}{\textit{arXiv} Template}


\hypersetup{
pdftitle={CLIPSE -- a minimalistic CLIP-based image search engine for research},
pdfsubject={image search engine},
pdfauthor={Steve Göring},
pdfkeywords={image search},
}

\begin{document}
\maketitle

\begin{abstract}

\end{abstract}


% keywords can be removed
\keywords{image search \and CLIP}


\section{Introduction}
Considering the increase of uploaded images per year, finding images to matching text descriptions is an ongoing challenge.
Various image search engines, e.g., Google search, Bing search, or open source engines (XY, XYZ), are available.
However, especially for research, small search engines considering own datasets are hard to find.
For this reason, a minimalistic search engine, called CLIPSE (CLIP-based image Search Engine), was developed.
CLIPSE is published as Open-Source software\footnote{\url{clipse}}.
The design follows a simplification approach, therefor everything is kept minimalistic.
This enables to perform a simple text based image search for a given data-set.



\section{Architecture Overview}
CLIPSE

uv

CLIP
flask
mvp


* build\_index.py: create an index based on a folder with images (recommended to downscale the images, e.g. to 480x480)
* query.py: command line interface to perform queries to an created index
* server.py: web interface

\section{Example}


\bibliographystyle{unsrtnat}
\bibliography{refs}



\end{document}
